%Dette er utf8x udgaven af LaTeX-templaten. Den er til brug på systemer der kører utf8x, såsom Linux. Hvis du bruger Windows, så er det letteste at bruge TeXmaker som editor.
% Loader dokumentklassen memoir. Sætter sproget i dokumentet til dansk, papirtypen til A4, sætter dokument til lige store højre og venstre margen, laver en søjler og siger at vi gerne vil lave en artikel, sætter skriftstørrelse til 11pt.
\documentclass[english,a4paper,oneside,onecolumn,oldfontcommands,article,11pt]{memoir}
\usepackage[danish]{babel}		%Giver mulighed for dansk orddeling. Slet kun hvis du VED hvad du laver, eller skal skrive noget på engelsk.
\usepackage[utf8x]{inputenc}	%Her sættes tegnsæt til utf8x.
\usepackage{graphicx}		%Tillader indsættelse af billeder
\usepackage{mathtools}		%Ekstra matematik... bare lad den være, du får muligvis brug for den.
\usepackage{url}		 %bruges til at formattere url'er... kan sagtens udelades.
\urlstyle{sf} % Url in same font as document (sans serif family)
\usepackage[colorlinks=false]{hyperref}
\usepackage[draft]{fixme} % Denne pakke tillader indsættelse af fixme noter ved brug af \fxnote{} som man kan bruge som huske sedler til sig selv.

\usepackage{listings} % Hvis man vil inkludere kode eksempler.
\lstset{language=matlab,breaklines=True} % Indstiller syntaks highlight til Matlab.

% Caption styles: Laver alle captions om til sans serif fonte og gør dem en smule mindre end brødteksten.
\captionnamefont{\small\sffamily}
\captiontitlefont{\small\sffamily}

\usepackage{soul} % lege lege -- Dette er til forsiden
\sodef\an{}{0.2em}{.9em plus.6em}{1em plus.1em minus.1em}
\newcommand\stext[1]{\an{\scshape#1}}


\usepackage{microtype} % Pakke der proever at fikse badbox problemer. Kun kompatibel med pdflatex.
% For at indstille margin:
%	   		         left   right ratio
\setlrmarginsandblock{2cm}{2cm}{*} % Hvor stjerne betyder udfyldes "automatisk".
%                     top    bot ratio
\setulmarginsandblock{2cm}{2cm}{*} % Hvor stjerne betyder udfyldes "automatisk".
%\setlength{\oddsidemargin}{0cm} %giver mere plads på siden
\checkandfixthelayout

\pagestyle{simple} %Giver tom footer og sidetal i header


%%%%%%%%%% MACROS %%%%%%%%%%
%% Engelsk matematik
\sisetup{decimalsymbol=period}

% Pæne referencer med \cref
\Crefname{equation}{equation}{equations}
\Crefname{figure}{figure}{figures}

\usepackage[utf8x]{inputenc}	%Her sættes tegnsæt til utf8x.
\usepackage{graphicx}		    %Tillader indsættelse af billeder
\usepackage{mathtools}		    %Ekstra matematik

% Tabeller
\usepackage{multirow}		% Tillader kombination af rækker i tabeller
\usepackage{dcolumn}		%Bruges til at lave matematiske tabelsøjler... se datatabel
\usepackage{booktabs}		%linjer i tabeller...
\usepackage{subcaption}		% Tillader caption til "subfigures"
\usepackage[font={small,sl}]{caption}	% Caption med skrå tekst ikke kursiv

\usepackage{wrapfig}		% Tillader ombrydning af tekst ift. figurer


% Links til internetsider
\usepackage{url} % Links skrives "\url{<link>}"
\newcommand{\furl}[1]{\footnote{ \url{#1}}}	% Links i fodnoter skrives "\furl{<link>}"

% Matematikpakker
\usepackage{amsmath,amssymb}

% Enheder
\usepackage[expproduct = cdot]{siunitx}	        %Bruges \SI{<tal>}{<enhed>}, \si{<enhed>} eller \num{<tal>}.
\sisetup{separate-uncertainty = true}	% Usikkerheder står som (<x> \pm <sig_x>)<enhed>
\sisetup{range-units = brackets, range-phrase = --}		% Intervaller med enheder som (<x_min> - <xmax>)<enhed>
\DeclareSIUnit\year{yr}				% Enhed år, skrevet <yr>, defineres som \year

% Matematisk notation
\newcommand{\dak}{^\dagger}	% Hermitisk konjugering skrives "\dak"

% Lighedstegn med ovenstående betingelser som \xleq{<betingelse>}
\usepackage{extarrows}
\newcommand{\xleq}{\xlongequal}

% Pæne græske bogstaver
\renewcommand{\epsilon}{\varepsilon}
\renewcommand{\phi}{\varphi}

% Matematiske mængder
\newcommand{\N}{\ensuremath{\mathbb{N}}} % Naturlige tal
\newcommand{\Z}{\ensuremath{\mathbb{Z}}} % Hele tal
\newcommand{\Q}{\ensuremath{\mathbb{Q}}} % Rationelle tal
\newcommand{\R}{\ensuremath{\mathbb{R}}} % Reelle tal
\newcommand{\C}{\ensuremath{\mathbb{C}}} % Komplekse tal
\newcommand{\F}{\ensuremath{\mathbb{F}}} % Legeme tal
\newcommand{\A}{\ensuremath{\mathbb{A}}} % Algebraiske tal

% Differentialer
% Differentieres f mht. x n gange skrive \(p)dif[n]{f}{x}, hvor p giver bløde afledede. For n=1 lades parentesen være tom, [].
\renewcommand*\d{\mathrm{d}}	% Ikke-kursiverede d'er som \d
\newcommand{\dr}{\d r}			% dr til f.eks. integraler som \dr
\newcommand{\dx}{\d x}			% dx til f.eks. integraler som \dx
%haard differentiering
\newcommand{\dif}[3][]{\frac{\d^{#1}{#3}}{{\d {#2}}^{#1}}}

%partiel differentiering
\newcommand{\pdif}[3][]{\frac{\partial^{#1}{#3}}{\partial {#2}^{#1}}}

\newcommand{\dt}[1]{\dot{#1}} % afledt mht. t (dot-notation)
\newcommand{\ddt}[1]{\ddot{#1}} % dbl.afledt mht. t (dbl.dot)

% Vektorer
\let\vaccent=\v % Omdøb \v{} til \vaccent{}
%\newcommand{\gv}[1]{{\vec{\mathbf{#1}}}} % Vektor med græske bogstaver
\renewcommand{\v}[1]{\mathbf{#1}} % Vektor med fed
\newcommand{\hatvec}[1]{\hat{\mathbf{#1}}}			% Hatvektor
\newcommand{\ihat}{\mathbf{\hat{\imath}}}		% Enhedsvektor i
\newcommand{\jhat}{\mathbf{\hat{\jmath}}}		% .. j
\newcommand{\khat}{\mathbf{\hat{k}}}			% .. k
\newcommand{\rhat}{\mathbf{\hat{r}}}			% .. r
\newcommand{\phihat}{\mathbf{\hat{\phi}}}			% .. phi
\newcommand{\thetahat}{\mathbf{\hat{\theta}}}			% .. theta
\newcommand{\shat}{\mathbf\boldsymbol{\hat{s}}}			% .. s
\newcommand{\xhat}{\mathbf{\hat{x}}}			% Enhedsvektor x
\newcommand{\yhat}{\mathbf{\hat{y}}}			% .. y
\newcommand{\zhat}{\mathbf{\hat{z}}}			% .. z
\newcommand{\grad}[1]{\gv{\nabla} #1} % Gradient
\let\divsymb=\div % Omdøb \div til \divsymb
\renewcommand{\div}[1]{\gv{\nabla} \cdot \v{#1}} % Divergens
\newcommand{\curl}[1]{\gv{\nabla} \times \v{#1}} % Curl
% Vil man tage div eller curl af græske bogstaver,
% skal man lade argumentetet være fx \gv{\mu} for µ-vektor


\newcommand{\e}[1]{\cdot 10^{#1}}					% *10^n ved \e{n}	
\newcommand{\inv}[1]{\dfrac{1}{#1}}					% 1/x ved \inv{x}
\newcommand{\invs}[1]{\dfrac{1}{\sqrt{#1}}}			% 1/sqrt{x} ved \invs{x}
\newcommand{\sto}{\sqrt{2}}							% sqrt{2} ved \sto
\newcommand{\stre}{\sqrt{3}}							% sqrt{3} ved \stre
\newcommand{\ha}{\inv{2}}							% ½ ved \ha

% Her omdefineres \exp så det inkl. paranteser, der tilpasser sig i størrelse med syntax \exp{...}
\let\oldexp\exp
\renewcommand{\exp}[1]{\oldexp\left( #1 \right)}

% Brøk med parenteser om, der selv tilpasser størrelse, som \pfrac{a}{b} for (a/b)
\newcommand{\pfrac}[2]{\left(\frac{#1}{#2}\right)}

% Kvantemekanik
\usepackage{braket} % Smart bra-ket notation
\newcommand{\ps}[1]{\psi_{#1}}						% psi_<kvantetal> som \ps{<kvantetal>}
\newcommand{\op}[1]{\hat #1} % operator

\newcommand{\expect}[2]{\Braket{#1|#2|#1}}			% Forventningsværdi af Q i tilstand a som \expect{a}{Q}
\newcommand{\hilbert}{\ensuremath{\mathcal{H}}}		% Hilbertrums H som \hilbert
\newcommand{\tr}[1]{\text{Tr}\left(#1\right)} % Trace
\newcommand{\ptr}[2]{\text{Tr}_{#1}\left(#2\right)} % Partial trace

% Spacing
\linespread{1.1}
\newcommand{\tablesize}[2]{
\setlength{\tabcolsep}{{#1} em}\def\arraystretch {#2}}	% \tablesize{x}{y} gør tabellen x gange bredere og y gange højere
\newcommand{\alignspace}[1]{\setlength{\jot}{#1}}	% \alignspace{n} gør environmentet align n gange højere
\newenvironment{nalign}{
    \begin{equation}
    \begin{aligned}
}{
    \end{aligned}
    \end{equation}
    \ignorespacesafterend
}
% \nalign giver midterstillet ligningsnummerering
\newenvironment{mtable}[1]{
\begin{center}
$\begin{array}{#1}}{\end{array}$\end{center}}		% \mtable gør hver celle til et math environment

% Referencer med parenteser omkring som \ref{ref1,ref2,...,refN}
\usepackage{cleveref}
\creflabelformat{equation}{#2(#1)#3}
\crefrangelabelformat{equation}{#3(#1)#4 to #5(#2)#6}
\renewcommand{\ref}[1]{\cref{#1}}

%Lad disse to linjer være. De sørger for at bunden af siden bliver pæn, og fjerner indryk ved afsnit.
\raggedbottom
\parindent = 0pt

% Differentierings d
\renewcommand{\d}{\mathrm{d}}

% Hård differentiering
\newcommand{\dif}[2][]{\frac{\d {#1}}{{\d {#2}}}}
\newcommand{\diff}[2][]{\frac{\d^{2}{#1}}{{\d {#2}}^2}}

% Partiel differentiering
\newcommand{\pdif}[2][]{\frac{\partial{#1}}{\partial {#2}}}
\newcommand{\pdiff}[2][]{\frac{\partial^2{#1}}{\partial {#2}^2}}

% Ligninger nummereres 1.1, 1.2 osv. i chapter 1 og 2.1, 2.2 osv. i chapter 2
\numberwithin{equation}{chapter}

% Ligninger nummereres 1.1.1, 1.1.2 osv. i section 1 og 2.1.1, 2.1.2 osv. i section 2 - dog også 1.0.1 og 2.0.1 i hhv. chapter 1 og 2.
%\numberwithin{equation}{section}

% phi defineres pænt
\renewcommand{\phi}{\varphi}

% Kvantemekanik
\newcommand{\op}[1]{\hat #1} % operator
\newcommand{\expect}[1]{\left< #1 \right>} % Forventningsværdi
\newcommand{\trace}{\ensuremath{\text{Tr}}\xspace}
\newcommand{\Hilbert}{\ensuremath{\mathcal{H}}}
\newcommand{\lag}{\ensuremath{{L}}}
\newcommand{\tr}[1]{\text{Tr}\left(#1\right)} % Trace
\newcommand{\ptr}[2]{\text{Tr}_{#1}\left(#2\right)} % Partial trace
\newcommand{\ket}[1]{\left| #1 \right>} % Dirac-notation: ket
\newcommand{\bra}[1]{\left< #1 \right|} % bra
\newcommand{\braket}[2]{\left< #1 \vphantom{#2} \, \right|
  \left. \! #2 \vphantom{#1} \right>} % bracket
\newcommand{\matrixel}[3]{\left< #1 \vphantom{#2#3} \right|
  #2 \left| #3 \vphantom{#1#2} \right>} % Bracket med ekstra streg
\newcommand{\dak}{^\dagger}	% Hermitisk konjugering skrives "\dak"

% Relativitetsteori
\newcommand{\co}[2]{{#1}_{#2}} % Covariant notation
\newcommand{\con}[2]{{#1}^{#2}} % Contravariant notation
\newcommand{\pco}[1]{\co{\partial}{#1}} % Covariant partialdiff
\newcommand{\pcon}[1]{\con{\partial}{#1}} % Contravariant partialdiff

% Vektor med fed skrift
\renewcommand{\v}[1]{\boldsymbol{#1}}
\newcommand{\vop}[1]{\v{\op{#1}}} % vector operator

% Enheder
\usepackage[expproduct = cdot]{siunitx}	        %Bruges \SI{<tal>}{<enhed>}, \si{<enhed>} eller \num{<tal>}.

% Pæne referencer med \cref
\usepackage{cleveref}
\crefname{equation}{equation}{equations}
\Crefname{equation}{Equation}{Equations}
\crefname{figure}{figure}{figures}
\Crefname{figure}{Figure}{Figures}
\crefname{chapter}{chapter}{chapters}
\Crefname{chapter}{Chapter}{Chapters}
\crefname{section}{section}{sections}
\Crefname{chapter}{Sections}{Sections}

%%%%%%%%%% MACROS ENDED %%%%%%%%%%



\begin{document}
%%%%%%%%%% FORSIDE %%%%%%%%%%

% UDKOMMENTER FOR DANSK FORSIDE
%%Her laver vi en flot forside...
%\begin{titlingpage}
%\thispagestyle{empty}
%\centering
%{ \setlength{\baselineskip}{24pt}
%{\Huge \stext{Kvantemekanikprojekt  E2018}} \par\vspace*{2\onelineskip}
%
%%%%%% HUSK AT ÆNDRE TITEL HER %%%%%%%%%%%%
%\large\stext{Projekt nr: 6} \par %<--- I stedet for X skriver du nr. på jeres projekt%%%%% HUSK AT ÆNDRE X HER %%%%%%%%%%%%
%\large\stext{Titel: Spin as a relativistic consequence} %<--- Her skriver du titlen på jeres projekt%%%%% HUSK AT ÆNDRE TITEL HER %%%%%%%%%%%%
%%%%%%%%%%%%%%%%%%%%%%%%%%%%%%%%%%%%%%%%%%%
%\par\vspace*{8\onelineskip} %Her skrives dit navn og dit årskortnummer
%\large\stext{Navn: Josefine Bj\o rndal Robl} \par
%\large\stext{Studienummer:  201706760}\par\vspace*{2\onelineskip}
%%% Skal flere gruppemedlemmer tilføjes, så kopier ovennstående to linjer, eller udkommenter og kopier nedenstående tre linjer ved individuel besvarelse.
%
%%\large\stext{Projektet er udf\o rt i samarbejde med:} \par
%%\large\stext{Navn: XX2 YY2 } \par
%%\large\stext{{\AA}rskortnummer:  123456789}\par\vspace*{2\onelineskip}
%
%\large\stext{Vi afleverer individuelle besvarelser}\par\vspace*{2\onelineskip}
%\large\stext{Afleveret: dato (Senest 7/12 kl.~12.00 til Ann-Berit Porse 
%St\ae rk\ae r, 1520-629) }\par\vspace*{2\onelineskip}
%}
%\vfill
%\vspace*{2\onelineskip}
%%%%% HER SKAL I SKRIVE VEJLEDEREN PÅ PROJEKT OG JERES TØ-INSTRUKTOR %%%%
%%%%% HVIS I HAR FORSKELLIGE INSTRKTORE TIL TØ, SÅ SKRIV DEM ALLE %%%%%%%
%\stext{Projektvejleder: Emilie Hindbo Clausen}\par\vspace*{2\onelineskip}
%\stext{Instruktor: Shaeema Zaman}\par\vspace*{2\onelineskip}
%\enlargethispage{2\onelineskip}
%\end{titlingpage}

% ENGELSK FORSIDE
%Her laver vi en flot forside...
\begin{titlingpage}
\thispagestyle{empty}
\centering
{ \setlength{\baselineskip}{24pt}
{\Huge \stext{Project in Quantum Mechanics E2018}} \par\vspace*{2\onelineskip}
\large\stext{Project no.: 6} \par
\large\stext{Title: Spin as a relativistic consequence}
\par\vspace*{8\onelineskip} %Her skrives dit navn og dit årskortnummer
\large\stext{Name: Josefine Bj\o rndal Robl} \par
\large\stext{Student number: 201706760}\par\vspace*{2\onelineskip}

\large\stext{The project is worked out in collaboration with:} \par
\large\stext{Navn: Rasmus Strid} \par
\large\stext{Student number: 201706621}\par\vspace*{2\onelineskip}
\large\stext{Navn: Peter Fl\o che Juelsgaard} \par
\large\stext{Student number: 201705593}\par\vspace*{2\onelineskip}

\large\stext{We hand in individual papers}\par\vspace*{2\onelineskip}
\large\stext{Handed in: December $7^{\text{th}}$ 2018}\par\vspace*{2\onelineskip}
}
\vfill\vspace*{2\onelineskip}
\stext{Project supervisor: Emilie Hindbo Clausen}\par\vspace*{2\onelineskip}
\stext{Instructor: Shaeema Zaman}\par\vspace*{2\onelineskip}
\enlargethispage{2\onelineskip}
\end{titlingpage}

%%%%%%%%%% FORSIDE SLUT %%%%%%%%%%


%%%%%%%%%% TEKST %%%%%%%%%%%

\chapter{Introduction}
Fundamental for quantum mechanics is the concept of particle-wave duality, expressed by the Einstein-de Broglie relations
\begin{align} \label{eq:EinsteinDeBroglieRelations}
	\v{p} = \hbar\v{k} \, \quad \text{and} \quad E = \hbar\omega \: ,
\end{align}
where $\v{p}$ and $E$ denotes the momentum and energy of the particle respectively, and $\v{k}$ and $\omega$ denotes the wave vector and the frequency of the wave respectively. Implied from these relations is that the particle must be describable by a wave function, which we demand contains the complete information of a particle's state (of motion) at a given time $t$, hence the wave function must satisfy a differential equation in time, which is the Schrödinger equation
\begin{align} \label{eq:OverordnetStrukturAfQMLigning}
	i\hbar\pdif{t}\psi(\v{r},t) &= \op{H}\psi(\v{r},t) \: ,
\end{align}
where $\op{H}$ is the Hamiltonian.
%
This, however, does not contain all the information we seek from a particle. We are missing a part of the puzzle: The spin of the particle. The complete description of a particle is postulated in \cite[chapter~4.4]{griffiths_introduction_2017} to be a combination of the spatial wave function, $\psi(\v{r},t)$, and a spinor, $\chi(\v{s},t)$,
\begin{align} \label{eq:TotalWaveFunctionWithSpinAndSpatialFunction}
	\Psi &= \psi(\v{r},t)\chi(\v{s},t) \: .
\end{align}
In the following a more general equation, the Dirac Equation, will be deduced, wherefrom the spin can be found as an integrated part of the particle's state.

\chapter{The Schrödinger equation} \label{chap:SchrodingerEquation}

Considering a non-relativistic particle with mass $m$ and momentum $\mathbf{p}$ in a potential $V$, the total energy can be found as
\begin{align} \label{eq:Schrodinger_ClassicalTotalEnergy}
	E &= T + V = \frac{\mathbf{p}^2}{2m} + V \: .
\end{align}
Using the operator representations of the energy and momentum,
\begin{align} \label{Schrodinger_ClassicalToQuantumMechanics_E_p}
	E \rightarrow i\hbar\pdif{t} \: , \quad
	\v{p} \rightarrow \frac{\hbar}{i}\boldsymbol\nabla \: ,
\end{align}
in \cref{eq:Schrodinger_ClassicalTotalEnergy} and letting it operate on the wave function of a particle, the equation becomes
\begin{align}
	i\hbar\pdif{t} \Psi(\v{r},t) &= -\frac{\hbar^2}{2m}\nabla^2\Psi(\v{r},t) + V\Psi(\v{r},t) \: ,
\end{align}
which is the Schrödinger equation, hence the Schrödinger equation originate from the non-relativistic equation of momentum, but this changes when in the relativistic realm, hence the energy also changes, and a relativistic Schrödinger-like equation is therefore needed to describe relativistic particles.

Considering now the same particle, but free from any potential, i.e. $V=0$, the solution is a generalisation to three dimensions of that found in \cite[eq.~2.94]{griffiths_introduction_2017},
\begin{align} \label{eq:SolutionFreeParticleSchrodinger}
	\Psi(\v{r},t) &= A\exp\left(i\left[\v{k}\cdot\v{r} - \frac{\hbar\v{k}^2}{2m}t\right]\right) = A\exp\left(\frac{i}{\hbar}\left[\v{p}\cdot\v{r} - Et\right]\right) \: ,
\end{align}
using the Einstein-de Broglie relations, \cref{eq:EinsteinDeBroglieRelations}, and the total energy of the particle is equal to kinetic energy of it due to the lack of potential.

Wanting to construct a relativistic Schrödinger-like equation one may first consider the structure of the Schrödinger equation itself. The Schrödinger equation is a differential equation of first order with respect to time and of second order with respect to position, but in special relativity, equations must be covariant, due to Einstein's principle of relativity - the laws of physics are the same in all frames of reference - hence time and space are treated equally. It is therefore reasonable to assume a relativistic Schrödinger-like equation to be a differential equation of either first or second order with respect to both time and position.

\chapter{The Klein-Gordon Equation}
The approach is to find a Schrödinger-like equation by replacing the non-relativistic energy with that of a relativistic particle, but otherwise do as normal. Considering a free particle with momentum $p = |\mathbf{p}|$ and invariant mass $m_0$, the relativistic energy is \cite[eq.~12.34]{uggerhoj_speciel_2016}
\begin{align} \label{eq:EnergyMomentumRelationRelativistic}
	E &= \sqrt{(\v{p}c)^2 + \left(m_0c^2\right)^2} \: .
\end{align}
A Hamiltonian with the above eigenenergy when applied to a momentum-eigenstate $\ket{\mathbf{p}}$ with eigenvalue $\mathbf{p}$ must be found, \cite[chap.~8.1]{sakurai_modern_2011}, but the square root turns out to be a problem in the early efforts to derive a relativistic equation, since the way of of dealing with the square root is via a Taylor expansion,
\begin{align}
	H &= \sqrt{\v{p}^2c^2 + m_0^2c^4}	
	= m_0c^2 \left(1+\frac{\v{p}^2}{m_0^2c^2}\right)^{\frac{1}{2}}
	= m_0c^2 + \frac{\v{p}^2}{2m_0} - \frac{\v{p}^4}{8m_0^3} + \frac{\v{p}^6}{16m_0^5} + \dots \: .
\end{align}
This, though, yields a non-covariant wave equation, since the time derivative from \cref{eq:OverordnetStrukturAfQMLigning} will be associated with different order spacial derivatives, which is not desirable.


\section{Derivation of the Klein-Gordon Equation}
To avoid these problems first attempts to find a relativistic wave equation used the square of the Hamiltonian, hence the square of the energy, instead of the Hamiltonian, and the energy, itself,
\begin{align} \label{eq:KleinGordon_EnergySquared}
	E^2 &= (\v{p}c)^2 + \left(m_0c^2\right)^2 \: .
\end{align}
Using the operator representations of $E$ and $\v{p}$ from \cref{Schrodinger_ClassicalToQuantumMechanics_E_p}, and letting it operate on the wave function of a particle \cref{eq:KleinGordon_EnergySquared} becomes
\begin{align}
	-\hbar^2 \pdiff{t}\Psi(\v{r},t) &= -\hbar^2c^2\nabla^2\Psi(\v{r},t) + m_0^2c^4\Psi(\v{r},t) \: ,
\end{align}
which is known as the Klein-Gordon equation.



\section{Properties of the Klein-Gordon equation}
The Klein-Gordon equation satisfy nearly all of the desirable qualities of a relativistic wave equation. Firstly it is covariant,\cite[p.~489]{sakurai_modern_2011}, i.e. independent of the frame of reference, which a relativistic equation shall be, as stated in \cref{chap:SchrodingerEquation}. Secondly the Klein-Gordon equation has solutions that are those expected for a free, relativistic particle of mass $m_0$, hence at the form (på formen af(?)) of \cref{eq:SolutionFreeParticleSchrodinger}. But the equation also comes with the downside of allowing negative probability density, $\rho = 2E|N|^2$ with $N$ denoting the normalization constant of $\Psi$, since the probability density is dependent on the energy, which are also allowed negative as a necessity for the solutions of the Klein-Gordon equation to form at complete set of basis states, \cite[p.~488]{sakurai_modern_2011}.

\chapter{The Dirac Equation}
%Most of the difficulties of the Klein-Gordon equation originate from it being a second order differential equation with respect to time. To avoid these issues Dirac tried, and in 1928 succeeded, in finding a first order differential equation with respect to time, now called the Dirac equation:
%\begin{align}
%	i\hbar\pdif{t}\Psi(\v{r},t) = \op{H}_D\Psi(\v{r},t) \: , \quad \op{H}_D = c\v{\alpha}\cdot\v{p} + m_0c^2\beta \: ,
%\end{align}
%where $\op{H}_D$ is the Dirac Hamiltonian.

Most of the difficulties of the Klein-Gordon equation originate from it being a second order differential equation with respect to time. To avoid these issues Dirac tried to, and in 1928 succeed in, finding a first order differential equation with respect to both time and space. This equation should satisfy the time-evolution equation of quantum mechanics, \cref{eq:OverordnetStrukturAfQMLigning}, and the Dirac Hamiltonian, $\op{H}_D$, since considering a free particle, would only depend on the fundamental constants $c$ and $\hbar$, the mass of the particle $m$ and the first order derivative with respect to the particle's coordinates, hence the Dirac equation is
\begin{align}
	i\hbar\pdif{t}\Psi(\v{r},t) = \op{H}_D\Psi(\v{r},t) \: , \quad \op{H}_D = c\v{\alpha}\cdot\v{p} + m_0c^2\beta \: ,
\end{align}
where $\v{\alpha}$ and $\beta$ must be dimensionless coefficients since $cp$(p som vektor(?)) and $m_0c^2$ have the dimensions of energy. Also they must be independent of $p_i$ to ensure a linear dependence of the spacial derivative, the momentum, and they must not depend on the coordinates, since this would introduce a force, which shall not be present for a free particle. These independences implies commutation between the momentum and the coefficients.

To gain further information about the coefficients, solutions to the Dirac equation is required to also satisfy the Klein-Gordon equation to secure correct relation between energy and momentum, hence the Dirac Hamiltonian squared must equal the Klein-Gordon Hamiltonian, since the Klein-Gordon equation yields $E^2\Psi = \op{H}_{KG}\Psi$ and the Dirac equation $E\Psi = \op{H}_D\Psi$.
\begin{align} \label{eq:KG=Dirac^2}
	c^2\v{p}^2 + m_0^2c^4 &= \left(c\v{\alpha}\cdot\v{p} + m_0c^2\beta\right)^2
	= c^2\left(\v{\alpha}\cdot\v{p}\right)^2 + m_0^2c^4\beta^2 + m_0c^3\left\{\left(\v{\alpha}\cdot\v{p}\right),\beta\right\} \: ,
\end{align}
where $\{\op{A},\op{B}\} = \op{A}\op{B} + \op{B}\op{A}$ is the anticommutator. From \cref{eq:KG=Dirac^2} it can be seen that $c^2\left(\v{\alpha}\cdot\v{p}\right)^2$ must yield $c^2\v{p}^2$ since it is the only term of second order in momentum and in light speed, $m_0^2c^4\beta^2$ must yield $m_0^2c^4$ since it is the only term of second order in mass and fourth order in light speed, hence $m_0c^3\left\{\left(\v{\alpha}\cdot\v{p}\right),\beta\right\} = 0$ since the Klein-Gordon equation does not contain momentum and mass of first order or light speed of third order.
Firstly it can be seen, that $\beta^2$ must be the identity since $m_0c^4 = m_0c^4\beta^2$. Secondly, writing the zero-term it can be seen that
\begin{align} \label{eq:0=anticommutatorOfAlpha_iAndBeta}
	0 &= \left\{\left(\v{\alpha}\cdot\v{p}\right),\beta\right\}
	= \left\{\sum_{i=1}^3 \alpha_i p_i,\beta\right\}
	= \sum_{i=1}^3 \alpha_i p_i \beta + \beta \alpha_i p_i
	= \sum_{i=1}^3 \left(\alpha_i \beta + \beta \alpha_i\right) p_i\:,
\end{align}
hence $0 = \left\{\alpha_i,\beta\right\}$ since \cref{eq:0=anticommutatorOfAlpha_iAndBeta} must be valid for an arbitrary $p_i$. Lastly
\begin{equation}
\begin{aligned}
	\v{p}^2 &= \left(\v{\alpha}\cdot\v{p}\right)^2
	= \sum_{i=1}^3 \alpha_i p_i \sum_{j=1}^3 \alpha_j p_j
	= \sum_{i=1}^3 \sum_{j=1}^3 p_i p_j \alpha_i \alpha_j
	= \sum_{i=1}^3 p_i p_j \alpha_i \alpha_j + p_j p_i \alpha_j \alpha_i \\
	&= \begin{cases}	
		\begin{array}{ll}
			\sum_{i=1}^3 2p_i^2\alpha_i^2 & \text{for } i = j \\
			\sum_{i=1}^3 p_i p_j \left\{\alpha_i\alpha_j\right\} & \text{for } i\ne j
		\end{array}
	\end{cases} \: ,
\end{aligned}
\end{equation}
hence $\left\{\alpha_i,\alpha_j\right\} = 2\delta_{ij}$ since $\v{p}^2$ only consist of the quadratic terms of $p_i$.
For $\Psi$ to satisfy both the Dirac equation and the Klein-Gordon equation we find that $\v{\alpha}$ and $\beta$ must satisfy
\begin{align}
	\left\{\alpha_i,\alpha_j\right\} = 2\delta_{ij} \: , \quad \left\{\alpha_i,\beta\right\} = 0 \: , \quad \text{and} \quad \beta^2 = 1 \: ,
\end{align}
which defines the Clifford algebra. (SOMETHING MORE HERE(?))

Another demand of the Dirac equation is that the Dirac Hamiltonian shall be hermitian since the energy is an observable. If $\op{H}_D$ is hermitian
\begin{equation} \label{eq:ShowAlphaAndBetaHermitian}
\begin{aligned}
	c\left(\v{\alpha}\cdot\v{p}\right) + \beta m_0c^2 = \op{H}_D
	&= \op{H}_D\dak = c\left(\v{\alpha}\cdot\v{p}\right)\dak + m_0 c^2\beta\dak \\
	\Rightarrow \quad c\left(\v{\alpha} - \v{\alpha}\dak\right)\cdot\v{p} &= m_0c^2\left(\beta\dak - \beta\right) \: ,
\end{aligned}
\end{equation}
since $\v{p}$ is an observable, $\v{p} = \v{p}\dak$. By the principle of relativity choosing the frame of reference for the particle, where $\v{p} = \v{0}$, to calculate $\beta$ is valid (OMFORMULERING???(?))
\begin{align}
	0 &= m_0c^2\left(\beta\dak - \beta\right) \quad
	\Rightarrow \quad \beta = \beta\dak \: ,
\end{align}
and using this to calculate $\v{\alpha}$ for an arbitrary $\v{p}$ yields
\begin{align}
	0 &= \quad c\left(\v{\alpha} - \v{\alpha}\dak\right)\cdot\v{p} \quad
	\Rightarrow \quad \v{\alpha} = \v{\alpha}\dak \: ,
\end{align}
hence both $\v{\alpha}$ and $\beta$ shall be hermitian for $\op{H}_D$ to be hermitian.

It turns out also to be important for the coefficients to be traceless, \cite{schlippe_qm07-08.pdf_2007}, with trace being the sum of the diagonal elements of a matrix, and these requirements along with linearly independence narrows the possible options to a few, that in the Dirac-Pauli representation are:
\begin{align}
	\alpha_i = \begin{pmatrix}
		0 & \sigma_i \\
		\sigma_i & 0 
	\end{pmatrix}
	\: , \quad \text{and} \quad
	\beta = \begin{pmatrix}
		I_2 & 0 \\
		0 & -I_2
	\end{pmatrix} \: ,
\end{align}
with $I_2$ being the identity matrix and $\sigma_i$ being the hermitian Pauli spin matrices \cite[eq.~3.2.32]{sakurai_modern_2011}
\begin{align}
	\sigma_1 = \begin{pmatrix}
		0 & 1 \\
		1 & 0
	\end{pmatrix}
	\: , \quad
	\sigma_2 = \begin{pmatrix}
		0 & -i \\
		i & 0
	\end{pmatrix}
	\: , \quad \text{and} \quad
	\sigma_3 = \begin{pmatrix}
		1 & 0 \\
		0 & -1
	\end{pmatrix} \: .
\end{align}
Per identities from \cite[p.~169]{sakurai_modern_2011} it can be realised, that $\v{\alpha}$ and $\beta$ satisfy the above stated requirements.



\section{Existence of the spin}
Tekst



\section{Solutions to the Dirac equation for a free particle}
TEKST

\chapter{Prediction of antimatter}

\section{The Dirac Sea}
A first attempt to explain the negative energy quantum states found predicted in the Dirac equation was the Dirac sea postulated by Dirac in 1930. The model makes use of the Pauli exclusion principle - that two ore more identical fermions can not occupy the same quantum state - to interpret the vacuum as an infinite sea of negatively charged particles, so that all negative-energy states are filled, but none of the positive-energy states.\cite{bian_deduction_2016} Introducing an electron the only available state for it to be placed is a positive-energy state, and should this electron emit photons and thereby lose energy, it is forbidden to drop to negative energy, hence ... (!!!something with that it then is how we would expect it to be!!!(?)). Occasionally a negative-energy particle could be supplied(?) with a sufficient amount of energy to be lifted out of the Dirac sea and become a particle of positive energy. This electron would leave behind a hole in the sea, that would act exactly like the positive-energy electron but with opposite charge. The process is known as pair production, where an subatomic particle and its antiparticle, here the electron and the hole, is created form a neutral boson, here a photon. These holes were by Dirac wrongly identified as protons, \cite[p.~363]{dirac_theory_1930}, and predicted the annihilation of these particles when the positive-energy electron drops into the hole and fills it up.

Dirac's wrongly interpretation increased confusion about the negative-energy interpretation of the Dirac equation and caused some difficulties(?). One of these problems, the annihilation, was addressed in 1930 by Oppenheimer, who showed that Dirac's proton and the electron would annihilate within $\SI{e-10}{\second}$, which implied the life time of a hydrogen atom to be $\SI{e-10}{\second}$, (but this was not consistent with earlier experimental results(?)). The other problem, the difference in mass of the proton and the electron, was addressed by Herman Weyl using the symmetry of charges in the Maxwell and Dirac equations, which showed the mass of Dirac's proton, the hole in the Dirac sea, to be equal to the mass of the electron.\cite[sec.~8]{rajasekaran_discovery_2003} This lead Dirac to (??? reform / rethink ???) his theory and in his paper \cite[p.~61]{dirac_quantised_1931} Dirac acknowledges the hole as being an anti-electron, which is now known as a positron.

SOMETHING ABOUT THE EXPERIMENTAL DISCOVERY???(?)\cite{anderson_positive_1933}


%%%%%%%%%% IDEAS TO EXTRA STUFF TO MIGHT BE ENCLUDED %%%%%%%%%%
%Still some problems arrise from this interpretation. Firstly the Dirac sea theory implied an infinite negative charge for the universe, but none is experimentally measured. Secondly the theory builds on the Pauli exclusion principle valid only for fermions, which means the theory won't work for other particles as bosons. Both of these problems are addressed and solved in a unified interpretation of antiparticles using quantum field theory.
%%%%%%%%%%%%%%%%%%%%%%%%%%%%%%%%%%%%%%%%%%%%%%%%%%%%%%%%%%%%%%%


\section{Modern interpretation of the Dirac equation}

%%%%%%%%%% TEKST SLUT %%%%%%%%%%


%%%%%%%%%% BIBLIOGRAFI %%%%%%%%%%
\newpage
\bibliographystyle{unsrt}
\bibliography{bibliography}
%%%%%%%%%% BIBLIOGRAFI SLUT %%%%%%%%%%


\end{document}


