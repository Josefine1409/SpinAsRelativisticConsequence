\chapter{Prediction of antimatter}

\section{The Dirac Sea}
A first attempt to explain the negative energy quantum states found predicted in the Dirac equation was the Dirac sea postulated by Dirac in 1930. The model makes use of the Pauli exclusion principle - that two ore more identical fermions can not occupy the same quantum state - to interpret the vacuum as an infinite sea of negatively charged particles, so that all negative-energy states are filled, but none of the positive-energy states.\cite{bian_deduction_2016} Introducing an electron the only available state for it to be placed is a positive-energy state, and should this electron emit photons and thereby lose energy, it is forbidden to drop to negative energy, hence ... (!!!something with that it then is how we would expect it to be!!!(?)). Occasionally a negative-energy particle could be supplied(?) with a sufficient amount of energy to be lifted out of the Dirac sea and become a particle of positive energy. This electron would leave behind a hole in the sea, that would act exactly like the positive-energy electron but with opposite charge. The process is known as pair production, where an subatomic particle and its antiparticle, here the electron and the hole, is created form a neutral boson, here a photon. These holes were by Dirac wrongly identified as protons, \cite[p.~363]{dirac_theory_1930}, and predicted the annihilation of these particles when the positive-energy electron drops into the hole and fills it up.

Dirac's wrongly interpretation increased confusion about the negative-energy interpretation of the Dirac equation and caused some difficulties(?). One of these problems, the annihilation, was addressed in 1930 by Oppenheimer, who showed that Dirac's proton and the electron would annihilate within $\SI{e-10}{\second}$, which implied the life time of a hydrogen atom to be $\SI{e-10}{\second}$, (but this was not consistent with earlier experimental results(?)). The other problem, the difference in mass of the proton and the electron, was addressed by Herman Weyl using the symmetry of charges in the Maxwell and Dirac equations, which showed the mass of Dirac's proton, the hole in the Dirac sea, to be equal to the mass of the electron.\cite[sec.~8]{rajasekaran_discovery_2003} This lead Dirac to (??? reform / rethink ???) his theory and in his paper \cite[p.~61]{dirac_quantised_1931} Dirac acknowledges the hole as being an anti-electron, which is now known as a positron.

SOMETHING ABOUT THE EXPERIMENTAL DISCOVERY???(?)\cite{anderson_positive_1933}


%%%%%%%%%% IDEAS TO EXTRA STUFF TO MIGHT BE ENCLUDED %%%%%%%%%%
%Still some problems arrise from this interpretation. Firstly the Dirac sea theory implied an infinite negative charge for the universe, but none is experimentally measured. Secondly the theory builds on the Pauli exclusion principle valid only for fermions, which means the theory won't work for other particles as bosons. Both of these problems are addressed and solved in a unified interpretation of antiparticles using quantum field theory.
%%%%%%%%%%%%%%%%%%%%%%%%%%%%%%%%%%%%%%%%%%%%%%%%%%%%%%%%%%%%%%%


\section{Modern interpretation of the Dirac equation}