\chapter{The Dirac Equation}
%Most of the difficulties of the Klein-Gordon equation originate from it being a second order differential equation with respect to time. To avoid these issues Dirac tried, and in 1928 succeeded, in finding a first order differential equation with respect to time, now called the Dirac equation:
%\begin{align}
%	i\hbar\pdif{t}\Psi(\v{r},t) = \op{H}_D\Psi(\v{r},t) \: , \quad \op{H}_D = c\v{\alpha}\cdot\v{p} + m_0c^2\beta \: ,
%\end{align}
%where $\op{H}_D$ is the Dirac Hamiltonian.

Most of the difficulties of the Klein-Gordon equation originate from it being a second order differential equation with respect to time. To avoid these issues Dirac tried to, and in 1928 succeed in, finding a first order differential equation with respect to both time and space. This equation should satisfy the time-evolution equation of quantum mechanics, \cref{eq:OverordnetStrukturAfQMLigning}, and the Dirac Hamiltonian, $\op{H}_D$, since considering a free particle, would only depend on the fundamental constants $c$ and $\hbar$, the mass of the particle $m$ and the first order derivative with respect to the particle's coordinates, hence the Dirac equation is
\begin{align}
	i\hbar\pdif{t}\Psi(\v{r},t) = \op{H}_D\Psi(\v{r},t) \: , \quad \op{H}_D = c\v{\alpha}\cdot\v{p} + m_0c^2\beta \: ,
\end{align}
where $\v{\alpha}$ and $\beta$ must be dimensionless coefficients since $cp$(p som vektor(?)) and $m_0c^2$ have the dimensions of energy. Also they must be independent of $p_i$ to ensure a linear dependence of the spacial derivative, the momentum, and they must not depend on the coordinates, since this would introduce a force, which shall not be present for a free particle. These independences implies commutation between the momentum and the coefficients.

To gain further information about the coefficients, solutions to the Dirac equation is required to also satisfy the Klein-Gordon equation to secure correct relation between energy and momentum, hence the Dirac Hamiltonian squared must equal the Klein-Gordon Hamiltonian, since the Klein-Gordon equation yields $E^2\Psi = \op{H}_{KG}\Psi$ and the Dirac equation $E\Psi = \op{H}_D\Psi$.
\begin{align} \label{eq:KG=Dirac^2}
	c^2\v{p}^2 + m_0^2c^4 &= \left(c\v{\alpha}\cdot\v{p} + m_0c^2\beta\right)^2
	= c^2\left(\v{\alpha}\cdot\v{p}\right)^2 + m_0^2c^4\beta^2 + m_0c^3\left\{\left(\v{\alpha}\cdot\v{p}\right),\beta\right\} \: ,
\end{align}
where $\{\op{A},\op{B}\} = \op{A}\op{B} + \op{B}\op{A}$ is the anticommutator. From \cref{eq:KG=Dirac^2} it can be seen that $c^2\left(\v{\alpha}\cdot\v{p}\right)^2$ must yield $c^2\v{p}^2$ since it is the only term of second order in momentum and in light speed, $m_0^2c^4\beta^2$ must yield $m_0^2c^4$ since it is the only term of second order in mass and fourth order in light speed, hence $m_0c^3\left\{\left(\v{\alpha}\cdot\v{p}\right),\beta\right\} = 0$ since the Klein-Gordon equation does not contain momentum and mass of first order or light speed of third order.
Firstly it can be seen, that $\beta^2$ must be the identity since $m_0c^4 = m_0c^4\beta^2$. Secondly, writing the zero-term it can be seen that
\begin{align} \label{eq:0=anticommutatorOfAlpha_iAndBeta}
	0 &= \left\{\left(\v{\alpha}\cdot\v{p}\right),\beta\right\}
	= \left\{\sum_{i=1}^3 \alpha_i p_i,\beta\right\}
	= \sum_{i=1}^3 \alpha_i p_i \beta + \beta \alpha_i p_i
	= \sum_{i=1}^3 \left(\alpha_i \beta + \beta \alpha_i\right) p_i\:,
\end{align}
hence $0 = \left\{\alpha_i,\beta\right\}$ since \cref{eq:0=anticommutatorOfAlpha_iAndBeta} must be valid for an arbitrary $p_i$. Lastly
\begin{equation}
\begin{aligned}
	\v{p}^2 &= \left(\v{\alpha}\cdot\v{p}\right)^2
	= \sum_{i=1}^3 \alpha_i p_i \sum_{j=1}^3 \alpha_j p_j
	= \sum_{i=1}^3 \sum_{j=1}^3 p_i p_j \alpha_i \alpha_j
	= \sum_{i=1}^3 p_i p_j \alpha_i \alpha_j + p_j p_i \alpha_j \alpha_i \\
	&= \begin{cases}	
		\begin{array}{ll}
			\sum_{i=1}^3 2p_i^2\alpha_i^2 & \text{for } i = j \\
			\sum_{i=1}^3 p_i p_j \left\{\alpha_i\alpha_j\right\} & \text{for } i\ne j
		\end{array}
	\end{cases} \: ,
\end{aligned}
\end{equation}
hence $\left\{\alpha_i,\alpha_j\right\} = 2\delta_{ij}$ since $\v{p}^2$ only consist of the quadratic terms of $p_i$.
For $\Psi$ to satisfy both the Dirac equation and the Klein-Gordon equation we find that $\v{\alpha}$ and $\beta$ must satisfy
\begin{align}
	\left\{\alpha_i,\alpha_j\right\} = 2\delta_{ij} \: , \quad \left\{\alpha_i,\beta\right\} = 0 \: , \quad \text{and} \quad \beta^2 = 1 \: ,
\end{align}
which defines the Clifford algebra. (SOMETHING MORE HERE(?))

Another demand of the Dirac equation is that the Dirac Hamiltonian shall be hermitian since the energy is an observable. If $\op{H}_D$ is hermitian
\begin{equation} \label{eq:ShowAlphaAndBetaHermitian}
\begin{aligned}
	c\left(\v{\alpha}\cdot\v{p}\right) + \beta m_0c^2 = \op{H}_D
	&= \op{H}_D\dak = c\left(\v{\alpha}\cdot\v{p}\right)\dak + m_0 c^2\beta\dak \\
	\Rightarrow \quad c\left(\v{\alpha} - \v{\alpha}\dak\right)\cdot\v{p} &= m_0c^2\left(\beta\dak - \beta\right) \: ,
\end{aligned}
\end{equation}
since $\v{p}$ is an observable, $\v{p} = \v{p}\dak$. By the principle of relativity choosing the frame of reference for the particle, where $\v{p} = \v{0}$, to calculate $\beta$ is valid (OMFORMULERING???(?))
\begin{align}
	0 &= m_0c^2\left(\beta\dak - \beta\right) \quad
	\Rightarrow \quad \beta = \beta\dak \: ,
\end{align}
and using this to calculate $\v{\alpha}$ for an arbitrary $\v{p}$ yields
\begin{align}
	0 &= \quad c\left(\v{\alpha} - \v{\alpha}\dak\right)\cdot\v{p} \quad
	\Rightarrow \quad \v{\alpha} = \v{\alpha}\dak \: ,
\end{align}
hence both $\v{\alpha}$ and $\beta$ shall be hermitian for $\op{H}_D$ to be hermitian.

It turns out also to be important for the coefficients to be traceless, \cite{schlippe_qm07-08.pdf_2007}, with trace being the sum of the diagonal elements of a matrix, and these requirements along with linearly independence narrows the possible options to a few, that in the Dirac-Pauli representation are:
\begin{align}
	\alpha_i = \begin{pmatrix}
		0 & \sigma_i \\
		\sigma_i & 0 
	\end{pmatrix}
	\: , \quad \text{and} \quad
	\beta = \begin{pmatrix}
		I_2 & 0 \\
		0 & -I_2
	\end{pmatrix} \: ,
\end{align}
with $I_2$ being the identity matrix and $\sigma_i$ being the hermitian Pauli spin matrices \cite[eq.~3.2.32]{sakurai_modern_2011}
\begin{align}
	\sigma_1 = \begin{pmatrix}
		0 & 1 \\
		1 & 0
	\end{pmatrix}
	\: , \quad
	\sigma_2 = \begin{pmatrix}
		0 & -i \\
		i & 0
	\end{pmatrix}
	\: , \quad \text{and} \quad
	\sigma_3 = \begin{pmatrix}
		1 & 0 \\
		0 & -1
	\end{pmatrix} \: .
\end{align}
Per identities from \cite[p.~169]{sakurai_modern_2011} it can be realised, that $\v{\alpha}$ and $\beta$ satisfy the above stated requirements.



\section{Existence of the spin}
Tekst



\section{Solutions to the Dirac equation for a free particle}
TEKST