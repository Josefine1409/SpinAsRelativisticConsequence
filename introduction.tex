\chapter{Introduction}
Fundamental for quantum mechanics is the concept of particle-wave duality, expressed by the Einstein-de Broglie relations, and implied (from this) is that the particle must be describable by a wave function. This wave function we demand contains the complete information of a particle's state (of motion) at a given time $t$, hence the wave function must satisfy a differential equation in time, which is the Schrödinger equation
\begin{align} \label{eq:OverordnetStrukturAfQMLigning}
	i\hbar\pdif{t}\psi(\v{r},t) &= \op{H}\psi(\v{r},t) \: ,
\end{align}
where $\op{H}$ is the Hamiltonian.
%
This, however, does not contain all the information we seek from a particle. We are missing a part of the puzzle: The spin of the particle. The complete description of a particle is postulated in \cite[chapter~4.4]{griffiths_introduction_2017} to be a combination of the spatial wave function, $\psi(\v{r},t)$, and a spinor, $\chi(\v{s},t)$,
\begin{align}
	\Psi &= \psi(\v{r},t)\chi(\v{s},t) \: .
\end{align}
In the following a more general equation, the Dirac Equation, will be deduced, wherefrom the spin can be found as an integrated part of the particle's state.