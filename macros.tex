% Engelsk matematik
\sisetup{decimalsymbol=period}

% Pæne referencer med \cref
\Crefname{equation}{equation}{equations}
\Crefname{figure}{figure}{figures}

\usepackage[utf8x]{inputenc}	%Her sættes tegnsæt til utf8x.
\usepackage{graphicx}		    %Tillader indsættelse af billeder
\usepackage{mathtools}		    %Ekstra matematik

% Tabeller
\usepackage{multirow}		% Tillader kombination af rækker i tabeller
\usepackage{dcolumn}		%Bruges til at lave matematiske tabelsøjler... se datatabel
\usepackage{booktabs}		%linjer i tabeller...
\usepackage{subcaption}		% Tillader caption til "subfigures"
\usepackage[font={small,sl}]{caption}	% Caption med skrå tekst ikke kursiv

\usepackage{wrapfig}		% Tillader ombrydning af tekst ift. figurer


% Links til internetsider
\usepackage{url} % Links skrives "\url{<link>}"
\newcommand{\furl}[1]{\footnote{ \url{#1}}}	% Links i fodnoter skrives "\furl{<link>}"

% Matematikpakker
\usepackage{amsmath,amssymb}

% Enheder
\usepackage[expproduct = cdot]{siunitx}	        %Bruges \SI{<tal>}{<enhed>}, \si{<enhed>} eller \num{<tal>}.
\sisetup{separate-uncertainty = true}	% Usikkerheder står som (<x> \pm <sig_x>)<enhed>
\sisetup{range-units = brackets, range-phrase = --}		% Intervaller med enheder som (<x_min> - <xmax>)<enhed>
\DeclareSIUnit\year{yr}				% Enhed år, skrevet <yr>, defineres som \year

% Matematisk notation
\newcommand{\dak}{^\dagger}	% Hermitisk konjugering skrives "\dak"

% Lighedstegn med ovenstående betingelser som \xleq{<betingelse>}
\usepackage{extarrows}
\newcommand{\xleq}{\xlongequal}

% Pæne græske bogstaver
\renewcommand{\epsilon}{\varepsilon}
\renewcommand{\phi}{\varphi}

% Matematiske mængder
\newcommand{\N}{\ensuremath{\mathbb{N}}} % Naturlige tal
\newcommand{\Z}{\ensuremath{\mathbb{Z}}} % Hele tal
\newcommand{\Q}{\ensuremath{\mathbb{Q}}} % Rationelle tal
\newcommand{\R}{\ensuremath{\mathbb{R}}} % Reelle tal
\newcommand{\C}{\ensuremath{\mathbb{C}}} % Komplekse tal
\newcommand{\F}{\ensuremath{\mathbb{F}}} % Legeme tal
\newcommand{\A}{\ensuremath{\mathbb{A}}} % Algebraiske tal

% Differentialer
% Differentieres f mht. x n gange skrive \(p)dif[n]{f}{x}, hvor p giver bløde afledede. For n=1 lades parentesen være tom, [].
\renewcommand*\d{\mathrm{d}}	% Ikke-kursiverede d'er som \d
\newcommand{\dr}{\d r}			% dr til f.eks. integraler som \dr
\newcommand{\dx}{\d x}			% dx til f.eks. integraler som \dx
%haard differentiering
\newcommand{\dif}[3][]{\frac{\d^{#1}{#3}}{{\d {#2}}^{#1}}}

%partiel differentiering
\newcommand{\pdif}[3][]{\frac{\partial^{#1}{#3}}{\partial {#2}^{#1}}}

\newcommand{\dt}[1]{\dot{#1}} % afledt mht. t (dot-notation)
\newcommand{\ddt}[1]{\ddot{#1}} % dbl.afledt mht. t (dbl.dot)

% Vektorer
\let\vaccent=\v % Omdøb \v{} til \vaccent{}
%\newcommand{\gv}[1]{{\vec{\mathbf{#1}}}} % Vektor med græske bogstaver
\renewcommand{\v}[1]{\mathbf{#1}} % Vektor med fed
\newcommand{\hatvec}[1]{\hat{\mathbf{#1}}}			% Hatvektor
\newcommand{\ihat}{\mathbf{\hat{\imath}}}		% Enhedsvektor i
\newcommand{\jhat}{\mathbf{\hat{\jmath}}}		% .. j
\newcommand{\khat}{\mathbf{\hat{k}}}			% .. k
\newcommand{\rhat}{\mathbf{\hat{r}}}			% .. r
\newcommand{\phihat}{\mathbf{\hat{\phi}}}			% .. phi
\newcommand{\thetahat}{\mathbf{\hat{\theta}}}			% .. theta
\newcommand{\shat}{\mathbf\boldsymbol{\hat{s}}}			% .. s
\newcommand{\xhat}{\mathbf{\hat{x}}}			% Enhedsvektor x
\newcommand{\yhat}{\mathbf{\hat{y}}}			% .. y
\newcommand{\zhat}{\mathbf{\hat{z}}}			% .. z
\newcommand{\grad}[1]{\gv{\nabla} #1} % Gradient
\let\divsymb=\div % Omdøb \div til \divsymb
\renewcommand{\div}[1]{\gv{\nabla} \cdot \v{#1}} % Divergens
\newcommand{\curl}[1]{\gv{\nabla} \times \v{#1}} % Curl
% Vil man tage div eller curl af græske bogstaver,
% skal man lade argumentetet være fx \gv{\mu} for µ-vektor


\newcommand{\e}[1]{\cdot 10^{#1}}					% *10^n ved \e{n}	
\newcommand{\inv}[1]{\dfrac{1}{#1}}					% 1/x ved \inv{x}
\newcommand{\invs}[1]{\dfrac{1}{\sqrt{#1}}}			% 1/sqrt{x} ved \invs{x}
\newcommand{\sto}{\sqrt{2}}							% sqrt{2} ved \sto
\newcommand{\stre}{\sqrt{3}}							% sqrt{3} ved \stre
\newcommand{\ha}{\inv{2}}							% ½ ved \ha

% Her omdefineres \exp så det inkl. paranteser, der tilpasser sig i størrelse med syntax \exp{...}
\let\oldexp\exp
\renewcommand{\exp}[1]{\oldexp\left( #1 \right)}

% Brøk med parenteser om, der selv tilpasser størrelse, som \pfrac{a}{b} for (a/b)
\newcommand{\pfrac}[2]{\left(\frac{#1}{#2}\right)}

% Kvantemekanik
\usepackage{braket} % Smart bra-ket notation
\newcommand{\ps}[1]{\psi_{#1}}						% psi_<kvantetal> som \ps{<kvantetal>}
\newcommand{\op}[1]{\hat #1} % operator

\newcommand{\expect}[2]{\Braket{#1|#2|#1}}			% Forventningsværdi af Q i tilstand a som \expect{a}{Q}
\newcommand{\hilbert}{\ensuremath{\mathcal{H}}}		% Hilbertrums H som \hilbert
\newcommand{\tr}[1]{\text{Tr}\left(#1\right)} % Trace
\newcommand{\ptr}[2]{\text{Tr}_{#1}\left(#2\right)} % Partial trace

% Spacing
\linespread{1.1}
\newcommand{\tablesize}[2]{
\setlength{\tabcolsep}{{#1} em}\def\arraystretch {#2}}	% \tablesize{x}{y} gør tabellen x gange bredere og y gange højere
\newcommand{\alignspace}[1]{\setlength{\jot}{#1}}	% \alignspace{n} gør environmentet align n gange højere
\newenvironment{nalign}{
    \begin{equation}
    \begin{aligned}
}{
    \end{aligned}
    \end{equation}
    \ignorespacesafterend
}
% \nalign giver midterstillet ligningsnummerering
\newenvironment{mtable}[1]{
\begin{center}
$\begin{array}{#1}}{\end{array}$\end{center}}		% \mtable gør hver celle til et math environment

% Referencer med parenteser omkring som \ref{ref1,ref2,...,refN}
\usepackage{cleveref}
\creflabelformat{equation}{#2(#1)#3}
\crefrangelabelformat{equation}{#3(#1)#4 to #5(#2)#6}
\renewcommand{\ref}[1]{\cref{#1}}

%Lad disse to linjer være. De sørger for at bunden af siden bliver pæn, og fjerner indryk ved afsnit.
\raggedbottom
\parindent = 0pt